\documentclass[letterpaper, 8pt]{extarticle}
\usepackage{amssymb,amsmath,amsthm,amsfonts}
\usepackage{multicol,multirow}
\usepackage{calc}
\usepackage{ifthen}
\usepackage[landscape]{geometry}
\usepackage[colorlinks=true,citecolor=blue,linkcolor=blue]{hyperref}

\usepackage{booktabs}
\usepackage{ulem}
\usepackage{enumitem}
\usepackage{tabulary}
\usepackage{graphicx}
\usepackage{siunitx}
\usepackage{tikz}
\usepackage{derivative}
\usepackage{svg}
\usepackage{listings}
\usepackage{setspace}
\usepackage{listings}
\usepackage{xcolor}
\usepackage{courier}
\usepackage{syntax}
\usepackage{mathpartir}
\usepackage{braket}

% minimal line spacing
% \setstretch{0.1}

% set borders (experimentally determined to minimize cutoff and maximize space on school printers)
\geometry{top=.25in,left=.25in,right=.25in,bottom=.35in}

% make figures work better in multicol
% \newenvironment{Figure}
% {\par\medskip\noindent\minipage}
% {\endminipage\par\medskip}

% \pagestyle{empty} % clear page

% rewrite section commands to be smaller
\makeatletter
\renewcommand{\section}{\@startsection{section}{1}{0mm}%
                                {-1explus -.5ex minus -.2ex}%
                                {0.5ex plus .2ex}%x
                                {\normalfont\normalsize\bfseries}}
\renewcommand{\subsection}{\@startsection{subsection}{2}{0mm}%
                                {-1explus -.5ex minus -.2ex}%
                                {0.5ex plus .2ex}%
                                {\normalfont\small\bfseries}}
\renewcommand{\subsubsection}{\@startsection{subsubsection}{3}{0mm}%
                                {-1ex plus -.5ex minus -.2ex}%
                                {1ex plus .2ex}%
                                {\normalfont\tiny\bfseries}}
\makeatother
\setcounter{secnumdepth}{0} % disable section numbering


% disable indenting
\setlength{\parindent}{0pt}
\setlength{\parskip}{0pt plus 0.5ex}

% Custom siunitx defs
\DeclareSIUnit\noop{\relax}
\NewDocumentCommand\prefixvalue{m}{%
\qty[prefix-mode=extract-exponent,print-unity-mantissa=false]{1}{#1\noop}
}

% Shorthand definitions
\newcommand{\To}{\Rightarrow}
\newcommand{\ttt}{\texttt}
\newcommand{\ra}{\rightarrow}

% condense itemize & enumerate
\let\olditemize=\itemize \let\endolditemize=\enditemize \renewenvironment{itemize}{\olditemize \itemsep0em}{\endolditemize}
\let\oldenumerate=\enumerate \let\endoldenumerate=\endenumerate \renewenvironment{enumerate}{\oldenumerate \itemsep0em}{\endoldenumerate}
\setlist[itemize]{noitemsep, topsep=0pt, leftmargin=*}
\setlist[enumerate]{noitemsep, topsep=0pt, leftmargin=*}

\title{3QI3}

\begin{document}
\raggedright
\tiny

% make listings look nicer
% \lstset{
%     tabsize = 2, %% set tab space width
%     showstringspaces = false, %% prevent space marking in strings, string is defined as the text that is generally printed directly to the console
%     basicstyle = \tiny\ttfamily, %% set listing font and size
%     breaklines = true, %% enable line breaking
%     numberstyle = \tiny,
%     postbreak = \mbox{\textcolor{red}{\(\hookrightarrow\)}\space}
% }

\begin{center}
    {\textbf{Physics 3QI3 - Key Exchange Edition}} \\
\end{center}
% set column spacing rules
\setlength{\premulticols}{1pt}
\setlength{\postmulticols}{1pt}
\setlength{\multicolsep}{1pt}
\setlength{\columnsep}{2pt}
\begin{multicols*}{5}
    \section{Linalg}
    Matrix Multiplication:
    \(
    \begin{pmatrix}
        a & b \\
        c & d
    \end{pmatrix}
    \begin{pmatrix}
        e & f \\
        g & h
    \end{pmatrix}
    =
    \begin{pmatrix}
        ae+bg & af+bh \\
        ce+dg & cf+dh
    \end{pmatrix}
    \)

    \textbf{Adjoint (Hermitian Conjugate):}
    \(A^\dagger = A^*\)
    (transpose the matrix and take the complex conjugate of each element)

    \textbf{Complex Conjugate:}
    Flip the sign of the imaginary part of a complex number

    \section{Classical Information Theory}
    Shannon Entropy/Information: \(H = -\sum p(a_i) \log p(a_i)\)

    \section{Thermodynamics}
    Gibbs Entropy: \(S = -k \sum p_i \log p_i\)

    \section{Communication Theory}
    \textbf{Shannon's Noiseless Coding Theorem:}
    For a given message,
    we only need \(N H(p)\) bits to encode it,
    where \(H(p) = -\sum p_i \log p_i\)
    \textbf{Shannon's Noisy Coding Theorem:}
    On average, we need at least \(\frac{N_0}{1-H(q)}\)
    bits to encode one of \(2^{N_0}\) equally probable messages,
    where \(H(q) = -[q \log q + (1-q) \log (1-q)]\)

    \section{Dirac Notation}
    \(\bra{\Psi} \Longleftrightarrow \ket{\psi}^\dagger\)

    \begin{tabular}{@{}lc@{}}\toprule
        Ket                        & Matrix                                                      \\ \midrule
        \(\ket{0}\) or \(\ket{H}\) & \(\begin{bmatrix} 1 \\ 0 \end{bmatrix}\)                    \\
        \(\ket{1}\) or \(\ket{V}\) & \(\begin{bmatrix} 0 \\ 1 \end{bmatrix}\)                    \\
        Diagonal Up                & \(\frac{1}{\sqrt{2}}\begin{bmatrix} 1 \\ 1 \end{bmatrix}\)  \\
        Diagonal Down              & \(\frac{1}{\sqrt{2}}\begin{bmatrix} 1 \\ -1 \end{bmatrix}\) \\
        Left Circular              & \(\frac{1}{\sqrt{2}}\begin{bmatrix} 1 \\ i \end{bmatrix}\)  \\
        Right Circular             & \(\frac{1}{\sqrt{2}}\begin{bmatrix} 1 \\ -i \end{bmatrix}\) \\
        \bottomrule
    \end{tabular}

    \includegraphics[width=\linewidth]{Bloch\_sphere.svg.png}

    \(\ket{\Psi} = \cos\frac{\theta}{2}\ket{0}+e^{i \phi}\sin\frac{\theta}{2}\ket{1}\)

    \begin{multicols*}{2}
        \(+x = \frac{1}{\sqrt{2}}(\ket{0}+\ket{1})\)
        \(-x = \frac{1}{\sqrt{2}}(\ket{0}-\ket{1})\)
        \(+y = \frac{1}{\sqrt{2}}(\ket{0}+i\ket{1})\)
        \(-y = \frac{1}{\sqrt{2}}(\ket{0}-i\ket{1})\)
    \end{multicols*}

    \textbf{Change of basis}
    Let \(\theta\) be a rotation of basis vectors,
    counterclockwise.

    \(\ket{x} = \cos\theta\ket{x'} - \sin\theta\ket{y'}\)
    and
    \(\ket{y} = \sin\theta\ket{x'} + \cos\theta\ket{y'}\)

    where \(\ket{x'}\) and \(\ket{y'}\) are the new basis vectors.

    \textbf{Outer Product}
    Given that
    \(\ket{\psi} = \)
    \(\ket{\psi}\bra{\phi} = \begin{bmatrix} \psi_1\phi_1 & \psi_1\phi_2 \\ \psi_2\phi_1 & \psi_2\phi_2 \end{bmatrix}\)

    % TODO: quantum state tomography

    \section{Operators}
    Operators produce another ket

    \textbf{Mean value of an observable}
    Measuring an observable
    \(\hat{V} = \sum_i v_i \ket{v_i}\bra{v_i}\)
    in the state \(\ket{\Psi}\)

    Obtains result \(v_i\) with probability
    \(p(v_i) = |\braket{v_i|\Psi}|^2\)

    Repeating measurement many times obtains \textbf{expectation value}
    \(\braket{V}=\sum_i P_i v_i = \sum_i |\braket{v_i|\Psi}|^2 v_i\)
    \(\braket{V}_\Psi = \braket{\Psi|\hat{V}|\Psi}\)

    \subsection{Uncertainty}
    Variance is
    \(
    \Delta V^2
    = \braket{\Psi|(\hat{V}-\braket{\Psi|\hat{V}|\Psi})^2|\Psi}
    \)
    \(
    \Delta V^2
    = \braket{\Psi|\hat{V}^2|\Psi} - \braket{\Psi|\hat{V}|\Psi}^2
    = \braket{\hat{V}^2}-\braket{\hat{V}}^2
    \)

    \subsection{Heisenberg Uncertainty Principle}
    \(\Delta x \Delta p \geq \frac{1}{2} |\braket{\psi | [\hat{A}, \hat{B}]| \psi}|\)
    (e.g. for \([\hat{x}, \hat{p}] = i \hbar\) we find \(\Delta x \Delta p \geq \frac{\hbar}{2}\))

    \subsection{Pauli Operators}

    \begin{tabular}{@{}l@{}} \toprule
        \(\hat{\sigma}_x
        = \begin{pmatrix}
              0 & 1 \\
              1 & 0
          \end{pmatrix}
        = \ket{0}\bra{1}+\ket{1}\bra{0}\)    \\
        \quad Eigenvectors: \(
        \begin{pmatrix}1\\0\end{pmatrix},
        \begin{pmatrix}0\\1\end{pmatrix}
        \)                                   \\
        \(\hat{\sigma}_y
        = \begin{pmatrix}
              0 & -i \\
              i & 0
          \end{pmatrix}
        = i(\ket{1}\bra{0}-\ket{0}\bra{1})\) \\
        \quad Eigenvectors: \(
        \frac{1}{\sqrt{2}}\begin{pmatrix}1\\i\end{pmatrix},
        \frac{1}{\sqrt{2}}\begin{pmatrix}1\\-i\end{pmatrix}
        \)                                   \\
        \(\hat{\sigma}_z
        = \begin{pmatrix}
              1 & 0  \\
              0 & -1
          \end{pmatrix}
        = \ket{0}\bra{0}-\ket{1}\bra{1}\)    \\
        \quad Eigenvectors: \(
        \frac{1}{\sqrt{2}}\begin{pmatrix}1\\1\end{pmatrix},
        \frac{1}{\sqrt{2}}\begin{pmatrix}1\\-1\end{pmatrix}
        \)                                   \\
        \(\hat{I}
        = \begin{pmatrix}
              1 & 0 \\
              0 & 1
          \end{pmatrix}
        = \ket{0}\bra{0}+\ket{1}\bra{1}\)    \\
        \quad Eigenvectors: \(
        \begin{pmatrix}0\\1\end{pmatrix},
        \begin{pmatrix}1\\0\end{pmatrix}
        \)                                   \\
        \bottomrule
    \end{tabular}

    (All have respective eigenvalues of +1 and -1)

    \subsubsection{Commutaton Relations}
    \begin{multicols*}{2}
        \begin{align*}
            [\hat{\sigma}_x, \hat{\sigma}_y] & = 2i\hat{\sigma}_z \\
            [\hat{\sigma}_y, \hat{\sigma}_z] & = 2i\hat{\sigma}_x \\
            [\hat{\sigma}_z, \hat{\sigma}_x] & = 2i\hat{\sigma}_y
        \end{align*}
        \begin{align*}
            \{\hat{\sigma}_x, \hat{\sigma}_y\} & = 0 \\
            \{\hat{\sigma}_y, \hat{\sigma}_z\} & = 0 \\
            \{\hat{\sigma}_z, \hat{\sigma}_x\} & = 0
        \end{align*}
    \end{multicols*}
    \([\hat{\sigma}_a, \hat{\sigma}_b]=2 i \epsilon_{abc} \hat{\sigma}_c\)

    For direction \(\vec{n}\),
    \(
    \vec{n} \cdot \vec{\hat{\sigma}}
    = n_x \hat{\sigma}_x
    + n_y \hat(\sigma)_y
    + n_z \hat{\sigma}_z
    \)

    For any operator,
    \begin{align*}
        \hat{H} & = \begin{pmatrix}
                        a      & c - id \\
                        c + id & b
                    \end{pmatrix}                                                                                 \\
                & = \frac{a+b}{2}\hat{\mathbb{I}} + \frac{a-b}{2}\hat{\sigma}_z + c\hat{\sigma}_x + d\hat{\sigma}_y
    \end{align*}

    \subsection{Common Gates}
    \textbf{Hadamard gate:}
    \(\hat{H} = \frac{1}{\sqrt{2}} \begin{pmatrix}
        1 & 1  \\
        1 & -1
    \end{pmatrix}
    = \frac{1}{\sqrt{2}} (\hat{\sigma_z} + \hat{\sigma_x})\)
    \textbf{Rotation operator:}
    \(\hat{R}(\vec{n}, \theta) = e^{-i \theta \vec{n} \cdot \vec{\hat{J}}}\)
    Where \(\vec{\hat{J}}\) is the angular momentum operator,
    and \(\vec{n} = (\sin\theta\cos\phi, \sin\theta\sin\phi, \cos\theta)\)
    is a unit vector.

    For spin-1/2, \(\vec{\hat{J}} = \frac{1}{2}\vec{\hat{\sigma}}\)

    \section{Tensor Products}
    Given that \(\ket{\psi} = \begin{pmatrix} a \\ b \end{pmatrix}\)
    and \(\ket{\phi} = \begin{pmatrix} c \\ d \end{pmatrix}\)
    \[
        \ket{\psi} \otimes \ket{\phi}
        = \begin{pmatrix}
            a\begin{pmatrix} c \\ d \end{pmatrix} \\
            b\begin{pmatrix} c \\ d \end{pmatrix}
        \end{pmatrix}
        = \begin{pmatrix}
            ac \\
            ad \\
            bc \\
            bd
        \end{pmatrix}
    \]

    For operators,
    \begin{align*}
        \hat{A} \otimes \hat{B}
         & = \begin{pmatrix}
                 a & b \\
                 c & d
             \end{pmatrix}
        \otimes
        \begin{pmatrix}
            \alpha & \beta  \\
            \gamma & \delta
        \end{pmatrix}                            \\
         & = \begin{pmatrix}
                 a\begin{pmatrix}
                 \alpha & \beta  \\
                 \gamma & \delta
             \end{pmatrix} &
                 b\begin{pmatrix}
                 \alpha & \beta  \\
                 \gamma & \delta
             \end{pmatrix} \\
                 c\begin{pmatrix}
                 \alpha & \beta  \\
                 \gamma & \delta
             \end{pmatrix} &
                 d\begin{pmatrix}
                 \alpha & \beta  \\
                 \gamma & \delta
             \end{pmatrix}
             \end{pmatrix}         \\
         & = \begin{pmatrix}
                 a\alpha & a\beta  & b\alpha & b\beta  \\
                 a\gamma & a\delta & b\gamma & b\delta \\
                 c\alpha & c\beta  & d\alpha & d\beta  \\
                 c\gamma & c\delta & d\gamma & d\delta
             \end{pmatrix}
    \end{align*}

    \subsection{Properties}
    Not commutative.
    Distributive:
    \(\ket{\psi}\otimes(\ket{\phi}+\ket{\varphi}) = \ket{\psi}\otimes\ket{\phi}+\ket{\psi}\otimes\ket{\varphi}\)
    \(\hat{A}\otimes(\hat{B}+\hat{C}) = \hat{A}\otimes\hat{B}+\hat{A}\otimes\hat{C}\)

    Operators can act on one photon and not the other:
    Eg, let
    \[
        \sigma_A^x
        = \begin{pmatrix}
            0 & 0 & 1 & 0 \\
            0 & 0 & 0 & 1 \\
            1 & 0 & 0 & 0 \\
            0 & 1 & 0 & 0
        \end{pmatrix}
    \]
    thus,
    \begin{align*}
        \sigma_A^x \ket{HH}
         & = \sigma_A^x \otimes \mathcal{I} (\ket{H}_A \otimes \ket{H}_B) \\
         & = (\sigma_A^x \ket{H}_A) \otimes (\mathcal{I} \ket{H}_B)       \\
         & = \ket{V}_A \otimes \ket{H}_B                                  \\
         & = \ket{VH}
    \end{align*}
    or
    \[
        \begin{pmatrix}
            0 & 0 & 1 & 0 \\
            0 & 0 & 0 & 1 \\
            1 & 0 & 0 & 0 \\
            0 & 1 & 0 & 0
        \end{pmatrix}
        \begin{pmatrix}
            1 \\
            0 \\
            0 \\
            0
        \end{pmatrix}
        =
        \begin{pmatrix}
            0 \\
            0 \\
            1 \\
            0
        \end{pmatrix}
    \]

    \section{Classical Cryptography}
    \textbf{Criterion for Perfect Secrecy}
    \(
    P(p_i | C_j) = P(p_i) \forall i, j
    \)

\end{multicols*}

\end{document}
